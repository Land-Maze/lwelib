\chapter*{Introduction}
\addcontentsline{toc}{chapter}{Introduction}
\markboth{Introduction}{Introduction}

The transition towards PQC (Post-Quantum Cryptography) is driven by the potential of quantum computers to break traditional asymmetric encryption.
Standard algorithms like RSA and ECC are vulnerable to Shor's algorithm, which solves the underlying mathematical problems in polynomial time.
In response, the NIST (National Institute of Standards and Technology) has spearheaded a process to standardize algorithms resistant to quantum attacks.

Lattice-based cryptography is a primary focus of this standardization, particularly schemes based on the Learning With Errors (LWE) problem.
While LWE offers high security, its implementation in practical systems is often hindered by large memory requirements $O(n^2 \log q)$ for keys, contributing to kilobyte-sized public keys for modest security levels.
The Module-LWE (M-LWE) variant addresses these limitations by utilizing structures (such as polynomial rings) that allow for smaller keys and faster computations while maintaining similar security levels to regular LWE\@.

This thesis explores the optimization and fast implementation of M-LWE\@.
We specifically focus on the mathematical structures used in the CRYSTALS-Kyber algorithm, analyzing how the NTT (Number Theoretic Transform) and vectorized instructions (SIMD Instructions) can be leveraged to achieve high-speed execution.
By optimizing these core components, we aim to demonstrate that M-LWE provides a robust and efficient solution for secure digital communication in the post-quantum era.

The goal of thesis is to implement and optimize the core operations of M-LWE based cryptographic schemes, with a focus on the CRYSTALS-Kyber algorithm.
We will analyze the mathematical foundations, implement efficient algorithms, and evaluate performance on modern hardware.

% FIXME: Add brief description about the structure of thesis